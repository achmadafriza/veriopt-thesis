\chapter{Conclusion \label{sec:conclusion}}

In order to provide solver-based tools for identifying incorrect optimizations, we introduced VeriTest: a software interface 
for automatic verification of GraalVM's expression optimizations. The main goal of this project is to integrate VeriTest into 
the GraalVM development workflow to assist developers in verifying proposed optimization rules without requiring expertise in formal 
proof systems. Through leveraging Isabelle and their automated tools such as Sledgehammer, Nitpick, and Quickcheck, VeriTest demonstrated 
its capability to detect malformed and incorrect optimization rules, and provide automatic verification of optimization rules.

Through the design patterns and algorithms that VeriTest has implemented, it has demonstrated its capability to provide fast and comprehensive 
analysis for verifying optimization rules. Producer-consumer pattern and sufficient mutual exclusion locks implemented inside ensured efficient 
processing for Isabelle invocations without bottlenecks. Furthermore, VeriTest displayed its extensibility for future refinement through its use of 
bridge and facade design patterns -- abstracting Isabelle interactions and minimizing tight-coupling.

While the evaluation of VeriTest on Veriopt's current theory base and GraalVM's source code annotation suggests that VeriTest cannot replace 
manual verification, it has demonstrated its potential in automatically verifying optimization rules; implying that the percentage of 
optimization rules that \emph{can} be automatically proven will improve as the collection of theories grows. Limitations were identified, 
particularly in VeriTest's ability to find counterexamples, due to the inherent limitations of Nitpick and Quickcheck. Despite these challenges, 
the tool provided valuable insights to previous works done and acts as a starting point for further refinement and expansion.

\section{Possible Future Work \label{sec:future_work}}

Due to the limitations of VeriTest in finding counterexamples, a potential direction for VeriTest's development is to enhance 
its capabilities and address the limitations found. To implement a more robust method of finding counterexamples, VeriTest can be 
extended with an interpreter capable of statically analyzing the semantics of Veriopt's domain specific language and transform it into a more 
robust theory to analyze -- possibly lowering intermediate representation expressions into an equivalent value or binary operations -- 
similar to Figure \ref{fig:ConditionalEvaluation}. Such interpreter would also benefit Veriopt to provide a less ambiguous grammar for the 
domain specific language. However, certain expressions might have complex side-conditions, which the analysis would need to 
consider.

Additionally, Isabelle's Quickcheck and Nitpick could be customized to suit the needs of GraalVM's intermediate representation by extending 
Isabelle/Scala and Isabelle/ML. This would greatly enhance VeriTest's capabilities, while still utilizing the framework that VeriTest provided.

Furthermore, the theory base for GraalVM's intermediate representation should be extended to represent a more 
comprehensive baseline for verifying expression optimizations. Currently, theorems and lemmas inside Veriopt are created specifically for verifying 
one optimization rule. A more generalized and comprehensive theory base would immensely improve VeriTest's capabilities.

Analyzing the usage of incomplete and unproven lemmas could represent a future iteration for VeriTest. 
This could be done either by identifying the cause of bug for Isabelle's oracle inside Isabelle Server, or having 
a list of unproven lemmas that can be matched. This would allow VeriTest to enhance the comprehensiveness of its analysis.

Another key fact that VeriTest demonstrated is that it is a \emph{generalized} interface for interacting with Isabelle. As such, VeriTest 
can be extended to automatically verify different compilers, given a \emph{sufficient} formal specification for its intermediate representation.
Furthermore, the algorithms used in VeriTest to ensure parallel processing and mutual exclusion can be used by Isabelle to enhance Isabelle Server's 
capabilities and alleviate its current limitations.

To support VeriTest's integration towards GraalVM's development workflow, it is necessary for VeriTest to be catered towards 
\emph{how} the developers would interface with VeriTest. This can be in the form of syntax highlighting inside a code editor by extending 
VeriTest into a language server protocol, or through a web interface that can display the analysis for optimization rules in a more 
human-readable form.

Overall, while VeriTest has shown promising results, continued development is necessary to fully realize its potential as a 
comprehensive tool for automatic verification of compiler optimizations.