\chapter{Introduction \label{sec:introduction}}

While compilers are generally believed to be correct, unwanted errors do happen, especially in the optimization phases of a compiler. 
This is due to: the common pitfalls of the 
language it's written in are \emph{just} accepted by the community; edge cases of the language that are not well considered; or simply making a 
mistake inside the implementation \cite[Sec. 1.2]{CompilerOptimization}. Human mistakes are natural in human-made software. As such, it is 
critical to \emph{try} to minimize the intrinsic risks of error in compilers.

Minimizing the risks of error is non-trivial. A suite of testing mechanisms is needed to ensure the reliability of software. 
There are several ways to do this. For software, regression testing in the form of Unit, Integration, and System level tests are the 
industry standard ways for mitigating risks \cite{testing}. 

Such testing suites are ideal for software with human-understandable behavior. However, 
the behaviors of compilers itself are not exactly human-readable. As such, manually defining the obscure behaviors of compilers is tedious and 
time-consuming \cite{compcertVerification}. Another way to verify the behavior of compilers is to \textbf{formally} 
verify their specifications \cite{compcertVerification}.

\section{Objectives \& Scope}

The primary purpose of this project is to provide a framework for answering the 2\textsuperscript{nd} step of the compiler research thread proposed by 
Lopes and Regehr:

\begin{quote}
    "Implement solver-based tools for finding incorrect optimizations, not-weakest
    preconditions, groups of optimizations that either subsume each other or
    undo each other, etc." \cite[p. 5]{CompilerOptimization}.
\end{quote}

Hence, this project introduces VeriTest: a software interface for automatic verification of GraalVM's expression optimizations. VeriTest aims to provide 
fast analysis over proposed optimization rules -- written in Veriopt's GraalVM intermediate representation (IR) domain specific language 
(DSL) \cite[Sec. 3]{Term_Graph_Optimizations} -- to identify incorrect optimizations. GraalVM developers could integrate the analysis into 
their development workflow. This would make it easy for GraalVM developers to use the tool \emph{as you go}, 
without being a \emph{"proof expert"} on Isabelle.

\begin{figure}[h]
    \begin{enumerate}
        \item GraalVM developers write a proposed optimization proof in Veriopt's DSL.
        \item The optimization rule would be passed into VeriTest by the developers with the following goals:

              \begin{enumerate}
                \item Determine if it is verifiable, either a simple proof for correct rules or counterexample for incorrect rules;
                \item If VeriTest can't determine, optimization rules would be passed to \emph{"proof experts"}.
              \end{enumerate}
    \end{enumerate}
    \caption{Proposed development workflow for GraalVM developers}
    \label{fig:toolGoals}
\end{figure}

To achieve this, we intend to leverage automation tools built in interactive theorem provers such as Isabelle \cite{IsabelleHOL}, inspired by the 
previous works such as CompCert \cite{compcertVerification} and Alive \cite{AliveInLean,Alive2}; all of which integrate the theoretical aspects of 
formal verification into the practicality of using it in the industry. This project builds upon the previous works done to introduce formal semantics 
for GraalVM's \cite{graal} IR \cite{ATVA21_GraalVM_IR_Semantics, Term_Graph_Optimizations} implemented in Isabelle \cite{IsabelleHOL}.

However, verification that DSL matches the implementation of GraalVM would be out of the scope of this project. It would represent a thorough 
2nd step of the compiler verification research thread \cite[p. 5]{CompilerOptimization} -- similar to Alive2's \cite{Alive2} solution on 
LLVM's formal verification. Furthermore, extending Veriopt's DSL and Isabelle's 
automated tools is out of the scope of this project, considering the time constraints of this project and the significant amount of work needed to 
undertake the endeavour. All of which could represent a future direction in the research.

Additionally, in relation to the evaluation of VeriTest, we will conduct benchmarks upon Veriopt's current theory base and GraalVM's source 
code annotations, which is elaborated in Section \ref{sec:results}. However, the purpose of this evaluation is not to benchmark Isabelle, 
but rather to provide a starting point to understand the capabilities of VeriTest. There is a community-led benchmark spearheaded by Huch and Bode 
\cite{huch_isabelle_2022} which evaluates the performance of Isabelle for typical computations.

\section{Thesis Structure}

In relation to the goals, there would be several questions implied:

\begin{enumerate}
    \item Is it possible to extend the current workings of Veriopt in Isabelle to a tool that could provide analysis for the developers?    
    \item Would the analysis be useful to GraalVM developers?
    \item How fast could the analysis be provided to GraalVM developers?
\end{enumerate}

In order to achieve the goals of this project, we conducted a literature review in Chapter \ref{sec:background} that explores 
why software testing is considered to be inadequate for eliminating errors inside a compiler, and the significancy of 
formally specifying a compiler -- done by CompCert \cite{compcertVerification}, Alive \cite{AliveInLean, Alive2}, and 
Veriopt \cite{ATVA21_GraalVM_IR_Semantics, Term_Graph_Optimizations}. Furthermore, the capabilities of Isabelle as an interactive 
theorem prover that can help provide analysis towards optimization rules are described in this chapter. Chapter \ref{sec:PossibleApproaches} 
examines the possible approaches to extend the current workings of Veriopt in order to analyze optimization rules.

Furthermore, Chapter \ref{sec:implementation} describes VeriTest's current implementation, following what we consider to be the 
best possible approach to analyze optimization rules considering the time constraints of this project. The implementation of VeriTest 
considers several key requirements: (1) extensibility and modularity of the implementation for future work to be done; (2) comprehensiveness 
of the analysis; and (3) performance of VeriTest as a system. Chapter \ref{sec:results} evaluates the key requirements of VeriTest, 
and examines our findings towards leveraging Isabelle as VeriTest's proof engine.

Finally, Chapter \ref{sec:conclusion} concludes by discussing the projects contribution to the field of compiler research. This chapter also 
reviews our findings and provides recommendations for future work to be done.