\chapter{Results \& Discussion \label{sec:results}}

In order to determine the thoroughness of VeriTest's analysis, we evaluated VeriTest based on Veriopt's current, \emph{proven}, collection of 
optimization rules inside their theory base \cite{Term_Graph_Optimizations}. Furthermore, as GraalVM's compiler source code is currently annotated 
with optimization rules following Veriopt's DSL \cite[Sec. 5.1]{Term_Graph_Optimizations}, we can evaluate VeriTest by evaluating the source 
code annotations through VeriTest.

Evaluation is done by exporting the existing Veriopt's expression canonicalization optimization rules using VeriTest's extraction tool 
(See Sec. \ref{sec:Tool}) and iterating them several times as a benchmark on an AMD Ryzen 9 7900X processor with 64GB of RAM inside a 
Windows 10 WSL environment.

\section{Evaluation of Veriopt's Current Theory Base}

\todo[inline]{add table from conference paper}

Our findings suggests that there is a baseline runtime for every verification of an optimization rule. This is because, surprisingly, there is a 
key flaw inside Isabelle Server. Kobschätzki \cite{kobschatzki_unexpected_2024} notes that Isabelle Server triggers a race condition with multiple 
users and sessions, unless a delay of several seconds is placed in between subsequent commands. Isabelle Server's source code also confirms that 
each command invocation doesn't support concurrent use. Since every command invocation is preceded -- and followed by -- session invocations, 
this flaw inside Isabelle Server presents a bottleneck for processing optimization rules.

\todo[inline]{describe the malformed rule and counterexample}
\todo[inline]{mention that the counterexample is for termination proof obligation, not the term refinement obligation}

The evaluated runtime suggests a significant variation of runtime for some results.
This is due to VeriTest attempting to exhaust every possible proof and counterexample for the optimization rule, as illustrated by section 
\ref{sec:generateAnalysis}. As the complexity of optimization rules vary, the depth of the recursive tree depends on the proof obligations that 
Sledgehammer can prove. Failure to find a proof suggests that it \emph{may} be able to be proven with adequate expertise in formal proofs.

\subsection{Incomplete Proofs}

While we do discover that VeriTest is capable of utilizing existing lemmas inside Veriopt's current theory base to find proofs for the 
optimization rule, we discovered that \emph{some} of the lemmas are in fact only partially proven and are still on progress. 
This introduces a moderate drawback for VeriTest. However, this can be remedied by adjusting VeriTest's current algorithm 
(See Fig. \ref{fig:sledgehammerPseudocode}) to consider Isabelle's usage of partially proven theories, and remove such proof if it 
exists (See appendix \ref{app:remediedPseudocode}).

\todo[inline]{add figure for number of optimization rules that still incorporates incomplete lemmas}

\todo[inline]{describe the figure}

\todo[inline]{add figure for results without incorporating incomplete proofs}

\todo[inline]{describe the figure}

\todo[inline]{tell that as the theory base grows, the percentage of proof should also grow}

\section{Evaluation of Malformed Rules}

In order to determine the completeness of VeriTest's analysis in finding incorrect optimization rules, we devised a suite of test cases describing 
malformed optimization rules. These malformed rules are generated through mutation operators devised by Offutt et al. \cite{offutt_mutation_2006}.
The mutated optimization rules should represent possible scenarios where a GraalVM developer might make mistakes in the syntax analogous to 
the Java syntax (i.e., using Logical Or (\mono{||}) instead of Bitwise Or (\mono{|})). The list of mutation operators applied and the mutated 
optimization rules can be seen in appendix \ref{app:mutatedOptRules}.

\todo[inline]{add figure for results for malformed rules}

\todo[inline]{describe the figure}

While VeriTest is capable of finding syntax and type unification errors, our investigation discovered that VeriTest is unable to 
discover a counterexample for the mutated optimization rules. While it may seem trivial intuitively, it does not appear to be so for Isabelle. 
While the reasons for it are inconclusive, and are in fact beyond the scope of this project, several factors that might explain why.

\subsection{Limitations for Finding Counterexamples}

\begin{figure}[!htb]
    \centering
    \(e_1~\sqsupseteq~e_2~=~(\forall~m~p~v.~[m,~p]~\vdash~e_1~\mapsto~v~\longrightarrow~~[m,~p]~\vdash~e_2~\mapsto~v)\)

    \caption{Term refinement proof obligation, Adapted from \cite[Definition 6]{Term_Graph_Optimizations}}
    \label{fig:RefinementProofObligation}
\end{figure}

Revisiting Quickcheck's limitations (See Sec. \ref{sec:IsabelleLimitations}), it is possible that Quickcheck's exhaustive test generator 
is unable to transform complex conjectures into datatypes that it can generate tests for, even when appropriate constructors are defined. 
For conditional conjectures for each optimization rule, such as figure \ref{fig:RefinementProofObligation}, Quickcheck may exhaust all 
possible values of IR expressions that define the context of \([m,~p]\) at the expression level, which it fails to do so.

\begin{figure}[!htb]
    \centering
    \(val[e_1]~\neq~\)\emph{UndefVal}\(~\land~val[e_2]~\neq~\)\emph{UndefVal}\(~\longrightarrow~val[e_1]~=~val[e_2]\)

    \caption{Conditional evaluation of \(e_1\) and \(e_2\)}
    \label{fig:ConditionalEvaluation}
\end{figure}

Instead of exhaustively enumerating all the possible contexts for an expression, it may be possible for Quickcheck to conditionally evaluate 
the given conjecture as a value, as illustrated in figure \ref{fig:ConditionalEvaluation}. As long as \(e_1\) and \(e_2\) are not undefined, 
the values of \(e_1\) and \(e_2\) are compared to be equal or not. This semi-automatic method of conditionally evaluating expression semantics 
would need VeriTest to be able to parse GraalVM's IR DSL and statically analyze and construct appropriate equations in order to verify it, 
which may represent potential future work to be done.

\todo[inline]{summary of what a quotient type is --> should i put this in the appendix?}

In the case of Nitpick, extensive debugging suggests that there may be an inherent flaw inside the encoding of GraalVM's expression semantics.
Inside Veriopt's DSL, the encoding for GraalVM's IR expressions is defined as functions and datatypes that would influence how Isabelle would 
reason about it. Such definitions would be used to translate optimization rules into conjunctions for SMT solvers to find satisfiability.

However, we found that among the translated conjunctions, especially for the term refinement proof obligation, 
Nitpick is unable to support \emph{representative functions} to properly map an Integer quotient type (See appendix \ref{app:quotientType}) 
into the respective first-order relational logic \cite[Ch. 8]{isabelleNitpick}. This is explicitly mentioned as a known bug inside Nitpick, 
and should be amended by defining a \emph{term postprocessor} that converts such quotient type into a standard mathematical notation 
\cite[Sec. 3.7]{isabelleNitpick} or correcting underspecified functions \cite[Ch. 8]{isabelleNitpick}. 
To quote from Blanchette \cite[Ch. 8]{isabelleNitpick}:

\begin{quote}
    "Axioms or definitions that restrict the possible values of the undefined
    constant or other partially specified built-in Isabelle constants (e.g.,
    \emph{Abs\_} and \emph{Rep\_} constants) are in general ignored. Again, such nonconservative extensions are generally considered bad style."
\end{quote}

\section{Evaluation of GraalVM's Source Code Annotations}

\todo[inline]{add table of graal annotation evaluation}

\lipsum[1]
