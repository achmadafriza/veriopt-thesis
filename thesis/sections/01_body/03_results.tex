\chapter{Results \& Discussion \label{sec:results}}

In order to determine the thoroughness of VeriTest's analysis, we evaluated VeriTest based on Veriopt's current, \emph{proven}, collection of 
optimization rules inside their theory base \cite{Term_Graph_Optimizations}. Furthermore, as GraalVM's compiler source code is currently annotated 
with optimization rules following Veriopt's DSL \cite[Sec. 5.1]{Term_Graph_Optimizations}, we can evaluate VeriTest by evaluating the source 
code annotations through VeriTest.

Evaluation is done by exporting the existing Veriopt's expression canonicalization optimization rules using VeriTest's extraction tool 
(See Sec. \ref{sec:Tool}) and iterating them several times as a benchmark on an AMD Ryzen 9 7900X processor with 64GB of RAM inside a 
Windows 10 WSL environment.

\section{Evaluation of Veriopt's Current Theory Base}

\begin{table}[!htb]
    \centering
    \begin{tabular}{llll}
      \toprule
      Result & \#Rules & Mean \pm\ \textit{SD} \\
      \midrule
      Failed & 59 & 87.33 \pm\ 49.80 \\
      Found Auto Proof & 7 & 37.73 \pm\ 3.92 \\
      Found Proof & 32 & 82.91 \pm\ 29.43 \\
      Found Counterexample & 1 & 40.79 \pm\ 4.02 \\
      Malformed & 13 & 37.96 \pm\ 4.02 \\
      No Subgoal & 2 & 37.93 \pm\ 3.38 \\
      \bottomrule
    \end{tabular}
    \caption{Evaluation of each existing Veriopt optimization rules based on runtime (in seconds)}
    \label{tab:evaluation}
\end{table}

Our findings suggest that there is a baseline runtime for every verification of an optimization rule. This is because, surprisingly, there is a 
key flaw inside Isabelle Server. Kobschätzki \cite{kobschatzki_unexpected_2024} notes that Isabelle Server triggers a race condition with multiple 
users and sessions, unless a delay of several seconds is placed in between subsequent commands. Isabelle Server's source code also confirms that 
each command invocation doesn't support concurrent use. Since every command invocation is preceded -- and followed by -- session invocations, 
this flaw inside Isabelle Server presents a bottleneck for processing optimization rules. On the condition that such flaw does not exist, 
the baseline runtime for verifying optimization rules could be reduced by approximately 24-36 seconds or possibly more -- depending 
on the depth of the recursion tree.

Surprisingly, some of the existing optimization rules are malformed. While it is a commented out rule -- and it is malformed due to a minor 
syntax error (i.e., \mono{\(\backslash\)mapsto} instead of \mono{\(\backslash\)longmapsto}) -- it is interesting to see that it has type unification errors over the IR 
expression. This modification is included inside our mutation tests, which is described in Section \ref{sec:evaluateMalformed}.

\begin{figure}[!htb]
    \textbf{optimization} \emph{SubNegativeConstant: \(x\ -\ (const\ (val[-y]))\ \longmapsto\ x\ +\ (const\ y)\)}
    \begin{enumerate}
        \item \(False\)
        \item \(BinaryExpr\ BinSub\ x\ (ConstantExpr\ (intval\_negate\ y))\ \sqsupseteq\ BinaryExpr\ BinAdd\ x\ (ConstantExpr\ y)\)
    \end{enumerate}

    \caption{Optimization rule that VeriTest found a counterexample}
    \label{fig:exampleCounterexample}
\end{figure}

A counterexample is found for one of the existing optimization rule, as illustrated in Figure \ref{fig:exampleCounterexample}. 
While it is also on a commented out rule, Isabelle is able to reason that the termination proof obligation would be simplified to 
\emph{False}. This is due to the fact that an expression termination is represented by a measure function, \emph{trm} of type 
\emph{IRExpr} \Rightarrow\ \emph{nat} \cite[Sec. 3.3]{Term_Graph_Optimizations}. \(val[e]\) effectively lowers an IR expression into an 
equivalent value. Isabelle can immediately reason that the size of an equivalent value of an expression \(-y\) is not equal to the expression \(y\). 
This implies that the optimization rule will not terminate on any context of an optimization phase.

The evaluated runtime suggests a significant variation of runtime for some results.
This is due to VeriTest attempting to exhaust every possible proof and counterexample for the optimization rule, as illustrated by Section 
\ref{sec:generateAnalysis}. As the complexity of optimization rules varies, the depth of the recursive tree depends on the proof obligations that 
Sledgehammer can prove. Failure to find a proof suggests that it \emph{may} be able to be proven with adequate expertise in formal proofs.

Finally, we found that VeriTest is capable of utilizing existing lemmas inside Veriopt's current theory base to find proofs for the 
optimization rule. While we need to consider if the usage of such lemmas are proper -- as it may use incomplete lemmas 
(See Section \ref{sec:incompleteProof}) -- this also implies that, as the collection of theories grows, the percentage of optimization 
rules that can be automatically proven is expected to improve.

\subsection{Incomplete Proofs}
\label{sec:incompleteProof}

We have discovered that \emph{some} of the lemmas are in fact only partially proven and are still on progress, due to the fact that 
Veriopt \cite{Term_Graph_Optimizations} are still in development of proving GraalVM's optimizations.
This introduces a moderate drawback for the goals of VeriTest, as incorporating incomplete lemmas would effectively lead to an incorrect 
optimization rule.

Using incomplete proofs to prove an optimization rule can be detected by utilizing oracles through \emph{thm\_oracles} 
\cite[Sec. 5.14]{isabelleIsar}. Oracles works by leveraging external reasoners to produce arbitrary Isabelle theorems treated as an axiom.
This is particularly useful as incomplete proofs would then invoke such oracles and mark the arbitrary Isabelle theorem used to skip proof 
obligations.

\begin{figure}[!htb]
    \centering
    \begin{alignat}{3}
        \text{\textbf{optimization} } & \text{\emph{AddNeutral}} & : \quad & (e~+~\text{\emph{const}}~(\text{\emph{IntVal}}~32~0))~\longmapsto~e & \nonumber \\
        & & & \text{\emph{using AddNeutral\_Exp by presburger}} & \nonumber \\
        \text{\textbf{thm\_oracles} } & \text{\emph{AddNeutral}} & & & \nonumber
    \end{alignat}

    \caption{Example of an Isabelle Oracle}
    \label{fig:isabelleOracle}
\end{figure}

However, we found that the behaviors of the oracle invoked directly inside Isabelle and inside Isabelle Server has discrepancies between them.
Verifying Figure \ref{fig:isabelleOracle} inside Isabelle would not invoke any external oracle, as it already proved all required proof obligations.
Even so, when invoked inside Isabelle Server, the oracle would produce an arbitrary Isabelle theorem and skip the necessary proof obligation.

While the reasons for this is inconclusive, this is possibly a bug inside Isabelle, tracing the exact reasons inside Isabelle would prove to be 
a challenge. As such, we were unable to determine which of the existing theorems verified by VeriTest are incorporating incorrect lemmas.

A possible remedy to the limitations of Isabelle Server would be to have a list of such unproven lemmas inside VeriTest so that 
any proofs found from VeriTest can be matched and excluded. However, building such list only serves as a band-aid for the problem, and should 
not be treated as the answer. Enhancing VeriTest to consider usage of incorrect lemmas would represent a future refinement to the tool.

\section{Evaluation of Malformed Rules}
\label{sec:evaluateMalformed}

\begin{table}[!htb]
    \centering
    \begin{tabular}{llll}
      \toprule
      Result & \#Rules & Mean \pm\ \textit{SD} \\
      \midrule
      Failed & 25 & 80.95 \pm\ 18.53 \\
      Found Proof & 3 & 78.17 \pm\ 16.20 \\
      Found Counterexample & 2 & 56.22 \pm\ 25.16 \\
      Malformed & 1 & 35.98 \pm\ 0 \\
      Type Errors & 14 & 33.47 \pm\ 4.82 \\
      \bottomrule
    \end{tabular}
    \caption{Evaluation of each mutated optimization rules based on runtime (in seconds)}
    \label{tab:evaluationMalformed}
\end{table}

In order to determine the completeness of VeriTest's analysis in finding incorrect optimization rules, we devised a suite of test cases describing 
malformed optimization rules. These malformed rules are generated through mutation operators devised by Offutt et al. \cite{offutt_mutation_2006}.
The mutated optimization rules should represent possible scenarios where a GraalVM developer might make mistakes in the syntax analogous to 
the Java syntax (i.e., using a short-circuiting logical or (\mono{||}) instead of bitwise or (\mono{|})). The list of mutation operators 
applied and the mutated optimization rules can be seen in Appendix \ref{app:mutatedOptRules}. The results for the evaluation can be seen in 
Table \ref{tab:evaluationMalformed}.

\begin{align}
    &\text{\textbf{optimization} \emph{condition\_bounds\_x\_1: \(((u~>~v)~?~x~:~y)~\longmapsto~x\)}} \label{eq:conditionBoundsX}
\end{align}

Peculiarly, only one mutated optimization rule is regarded as a malformed optimization rule, while others are blended into the classification 
of a complex rule -- due to the limitations of Nitpick and Quickcheck -- or rules that have type unification errors within them. An example of 
them are type errors caused by utilizing \(>\) and \(\ge\), as illustrated by (\ref{eq:conditionBoundsX}). Both of the operators do not exist 
inside GraalVM's IR, but do exist in Isabelle as an operator that returns a \emph{bool}, which cannot be transformed into an \emph{IRExpr}.

\begin{align}
    &\text{\textbf{optimization} \emph{EliminateRedundantFalse\_1: \(x~|~true~\longmapsto~x\)}} \label{eq:EliminateRedundantFalse} \\
    &\text{\textbf{optimization} \emph{EliminateRedundantFalse\_2: \(x~||~true~\longmapsto~x\)}} \label{eq:EliminateRedundantFalse2}
\end{align}

Furthermore, there are some ambiguity to the grammar of Veriopt's DSL, particularly with \emph{BinShortCircuitOr} (\mono{||}), 
\emph{BinOr} (\mono{|}) -- as exemplified by (\ref{eq:EliminateRedundantFalse}). (\ref{eq:EliminateRedundantFalse2}) should be regarded as 
a \emph{syntactically} correct optimization rule, but type errors exist due to Isabelle regarding \mono{||} as two \emph{BinOr} nodes.

\begin{align}
    \text{\textbf{optimization} \emph{flipX: }} &((x~eq~(const~(IntVal~32~1)))~? \label{eq:flipX} \\
    &(const~(IntVal~32~1))~:~(const~(IntVal~32~0))) \nonumber \\
    &~~~~\longmapsto~x~\oplus~(const~(IntVal~32~1)) \nonumber \\
                            &when~(x~=~const~(IntVal~32~0)~| \nonumber \\
                            &~~~~~~~~(x~=~const~(IntVal~32~1))) \nonumber
\end{align}

This ambiguity is further exacerbated by the use of \emph{const} keyword and \emph{ConstantExpr} node, as seen in (\ref{eq:flipX}). 
A \emph{const} keyword is just a syntactic sugar for a \emph{ConstantExpr}. However, it cannot be used interchangeably inside 
a precondition of an optimization rule. A translation function \(exp[]\) needs to be used to mark the expression as an \emph{IRExpr}.
Furthermore, it is not clear when the translation function needs to be used, as opposed to just adding parenthesis.

\begin{align}
    &\text{\textbf{optimization} \emph{MulNeutral: }} x~*~ConstExpr~(IntVal~b~1)~\longmapsto~x \label{eq:MulNeutral}
\end{align}

The ambiguity extends to the definitions of an \emph{IRExpr}. As \emph{ConstExpr} is not included in the definition of \emph{IRExpr}, 
Isabelle regarded \emph{ConstExpr} as a function of type \(Value~\Rightarrow~IRExpr\) as seen in (\ref{eq:MulNeutral}).
While this is syntactically correct, it would mean that the resulting IR expression would not conform to the semantics of GraalVM's IR.

These findings indicate that the syntax of Veriopt's DSL may be ambiguous and rather unclear. This represents an opportunity for future research to 
formalize the grammar of the DSL, possibly adding a parser for the DSL, to reduce the ambiguity so that it is easier for 
GraalVM's developers to adopt and integrate it into their development workflow.

\subsection{Limitations for Finding Counterexamples}

Although VeriTest is capable of finding syntax and type unification errors, our investigation discovered that VeriTest is unable to 
discover counterexamples for \emph{specifically} the term refinement proof obligation of the mutated optimization rules. 
Even though it may seem trivial intuitively, it does not appear to be so for Isabelle. While the reasons for it are inconclusive, 
and are in fact beyond the scope of this project, we discuss several factors that might explain \emph{why} this happens.

\begin{figure}[!htb]
    \centering
    \(e_1~\sqsupseteq~e_2~=~(\forall~m~p~v.~[m,~p]~\vdash~e_1~\mapsto~v~\longrightarrow~~[m,~p]~\vdash~e_2~\mapsto~v)\)

    \caption{Term refinement proof obligation, Adapted from \cite[Definition 6]{Term_Graph_Optimizations}}
    \label{fig:RefinementProofObligation}
\end{figure}

For termination proof obligation, it is trivial for Isabelle to reason with it due to working over natural numbers. However, 
for an optimization rule's term refinement proof obligation, illustrated by Figure \ref{fig:RefinementProofObligation}, it is 
significantly more complex. Isabelle needs to be able to transform the proof obligation into conjectures that it can reason with.

Revisiting Quickcheck's limitations (See Sec. \ref{sec:IsabelleLimitations}), it is possible that Quickcheck's exhaustive test generator 
is unable to transform complex conjectures into datatypes that it can generate tests for, even when appropriate constructors are defined. 
For conditional conjectures for each optimization rule, such as Figure \ref{fig:RefinementProofObligation}, Quickcheck exhausts all 
possible values of IR expressions that define the context of \([m,~p]\) at the expression level, which it fails to do so.

\begin{figure}[!htb]
    \centering
    \(val[e_1]~\neq~\)\emph{UndefVal}\(~\land~val[e_2]~\neq~\)\emph{UndefVal}\(~\longrightarrow~val[e_1]~=~val[e_2]\)

    \caption{Conditional evaluation of \(e_1\) and \(e_2\)}
    \label{fig:ConditionalEvaluation}
\end{figure}

Instead of exhaustively enumerating all the possible contexts for an expression, it may be possible for Quickcheck to conditionally evaluate 
the given conjecture as a value, as illustrated in Figure \ref{fig:ConditionalEvaluation}. As long as \(e_1\) and \(e_2\) are not undefined, 
the values of \(e_1\) and \(e_2\) are compared to determine equality. This semi-automatic method of conditionally evaluating expression 
semantics would need VeriTest to parse GraalVM's IR DSL, statically analyze and construct appropriate and equivalent 
theorems to verify it, which may represent potential future work.

In the case of Nitpick, extensive debugging suggests that there may be an inherent flaw inside the encoding of GraalVM's expression semantics.
Inside Veriopt's DSL, the encoding for GraalVM's IR expressions is defined as functions and datatypes that influence how Isabelle 
reasons about it. Such definitions are used to translate optimization rules into conjunctions for SMT solvers to find satisfiability.

However, we found that among the translated conjunctions, especially for the term refinement proof obligation, 
Nitpick is unable to support \emph{representative functions} to properly map an Integer quotient type \cite[Sec. 3.7]{isabelleNitpick} 
into the respective first-order relational logic \cite[Ch. 8]{isabelleNitpick}. This is explicitly mentioned as a known bug inside Nitpick, 
and should be amended by defining a \emph{term postprocessor} that converts such quotient type into a standard mathematical notation 
\cite[Sec. 3.7]{isabelleNitpick} or correcting underspecified functions \cite[Ch. 8]{isabelleNitpick}. 
To quote from Blanchette \cite[Ch. 8]{isabelleNitpick}:

\begin{quote}
    "Axioms or definitions that restrict the possible values of the undefined
    constant or other partially specified built-in Isabelle constants (e.g.,
    \emph{Abs\_} and \emph{Rep\_} constants) are in general ignored. Again, such nonconservative extensions are generally considered bad style."
\end{quote}

\section{Evaluation of GraalVM's Source Code Annotations}

\begin{table}[!htb]
    \centering
    \begin{tabular}{llll}
      \toprule
      Result & \#Rules & Mean \pm\ \textit{SD} \\
      \midrule
      Failed & 12 & 73.78 \pm\ 4.66 \\
      Found Auto Proof & 1 & 29.22 \pm\ 0 \\
      Found Proof & 19 & 62.19 \pm\ 25.20 \\
      Found Counterexample & 3 & 40.79 \pm\ 5.10 \\
      Malformed & 21 & 38.02 \pm\ 5.16 \\
      Type Error & 40 & 37.85 \pm\ 5.43 \\
      \bottomrule
    \end{tabular}
    \caption{Evaluation of each existing GraalVM source code annotation based on runtime (in seconds)}
    \label{tab:evaluationGraal}
\end{table}

While the evaluation of existing GraalVM source code annotations (See Table \ref{tab:evaluationGraal}) is in line with the evaluation of 
Veriopt's existing optimization rules, one thing that stood out is the presence of significantly more optimization rules that have type 
unification errors. This can be attributed to the factors discussed in Section \ref{sec:evaluateMalformed}, where it is noted that the 
grammar of Veriopt's DSL can be rather ambiguous. This would also prove to be a challenge for GraalVM's developers to effectively integrate 
the DSL into their development workflow.

Integration would mean that the capabilities of VeriTest are extended into a language server protocol (LSP) to provide 
feedback while processing source code annotations in the background (See Section \ref{sec:modelDrivenDevelopment}). However, extending 
VeriTest into an LSP would require the DSL's grammar to be formalized such that a typical parser would be able to construct an abstract 
syntax tree (AST) for the DSL. Building such an AST would also mean that VeriTest can be extended to incorporate some metrics of static analysis,
allowing for an intelligent evaluation by building equivalent theorems (i.e., Figure \ref{fig:ConditionalEvaluation}) 
that would allow Isabelle to reason better.

\section{Challenges}

\begin{table}[!htb]
    \centering
    \begin{tabular}{lll}
      \toprule
      File Extension & Total LoC \\
      \midrule
      .java (Java classes) & 1751 \\
      .md (Markdown documents) & 1446 \\
      .thy (Isabelle theory files) & 952 \\
      .pl (Perl scripts) & 60 \\
      .py (Python scripts) & 56 \\
      .sh (Bash scripts) & 52 \\
      \bottomrule
    \end{tabular}
    \caption{Total lines of code (LoC) written for VeriTest}
    \label{tab:locTable}
\end{table}

Although the volume of code written reflects the efforts invested in this project, as illustrated by Table \ref{tab:locTable}, it does not 
adequately capture the full extent of the effort expended. Much of the effort is put into experimenting and capturing the full extent 
of the capabilities of Isabelle Server \cite{isabelleSystem} and their viability as a solution to this project. As a result of our due diligence, 
we discovered several bugs in Isabelle and identified future considerations for Veriopt's DSL \cite{Term_Graph_Optimizations}.

Furthermore, exploring the viability of Isabelle Server also meant that exploring the source code of Isabelle is necessary, as 
Isabelle's documentation is not always the most descriptive. Isabelle's source code is complex and challenging to 
read, which highlights the considerable effort invested in this project. This also proved to be a challenge for packaging 
Isabelle into a Docker container (See Section \ref{sec:Containerization}), as the dependencies of Isabelle are not explicitly stated 
anywhere in the documentation.

Moreover, careful considerations and analysis are made to ensure that VeriTest implements a mutually exclusive and concurrent analysis of 
the optimization rule, ensuring freedom from starvation for each invocation (See Section \ref{sec:implementation}). An elegant and 
well-crafted implementation is not necessarily reflected in the sheer volume of code written, but rather in the simplicity and modularity 
of the solution.
