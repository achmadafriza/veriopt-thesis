\section{Introduction}
% From the Project Proposal marking sheet:
%   - Define your topic such that there is no doubt about the intended
%   coverage and contribution of the thesis.
%   - Include a project outline and clear statement of purpose. 
%   - Show substantial evidence of initiative.

While compilers are deemed to be deterministic, unwanted errors would sometimes happen. This is due to the fact that: common pitfalls of the 
language its written in are \emph{just} accepted by the community; edge cases of the language are not well considered; or by simply making a 
mistake inside the implementation \cite[Sec. 1.2]{CompilerOptimization}. Human mistakes are natural in human-made software. As such, it is 
critical to \emph{try} to minimize the intrinsic risks of error in compilers.

Minimizing the risks of error is non-trivial. A suite of testing mechanisms are needed in order to ensure the reliability of a software. 
There are several ways to do this. For softwares, regression testing in the form of Unit, Integration, and System level tests are the 
industry standard ways for mitigating risks \cite{testing}. Such testing suites are ideal for software with a human understandable behavior. However, 
the behaviors of compilers itself are not exactly human-readable. As such, manually defining the obscure behaviors of compilers are tedious and 
time consuming \cite{compcertVerification}.

Another way to verify the behavior of compilers is to \textbf{formally} specify them \cite{compcertVerification}. This project follows up on 
previous works done to introduce formal semantics for GraalVM's \cite{graal} Intermediate Representation (IR)
\cite{ATVA21_GraalVM_IR_Semantics, Term_Graph_Optimizations} implemented in Isabelle \cite{IsabelleHOL}. There are similar works that have been 
done, i.e. CompCert \cite{compcertVerification} and Alive \cite{AliveInLean,Alive2}; all of which integrates the theoretical aspects of formal 
verification into the practicality of using it in a production setting.

This project would attempt to bridge the subset of gap between the formal semantics of GraalVM and integrating it into GraalVM's test suite 
\cite{Term_Graph_Optimizations}, focusing on creating an Automated Testing Framework for GraalVM's optimization DSL. The framework would represent an 
automated test generation and, \emph{if possible}, automated simple proof generation. This would make it easy for GraalVM's developers to use 
the tool \emph{as you go}, without being a \emph{"proof expert"} on Isabelle.

To implement an Automated Testing Framework for GraalVM's optimization DSL, there are several options for the project to explore 
(in order of ideal solutions):
\begin{enumerate}
    \item Utilizing \mono{Isabelle Server} - \mono{Client} interactions \cite[Ch. 4]{isabelleSystem} to generate test suite and simple proofs 
          \cite{isabelleQuickcheck,isabelleProof,isabelleNitpick,isabelleSledgehammer};
    \item Extend the system of \mono{Isabelle/Scala} to utilize the full functionality of Isabelle \cite[Ch. 5]{isabelleSystem};
    \item Creating an interpreter for the DSL, and applying a set of rules as a test suite.
\end{enumerate}

However, verification that DSL matches the implementation of GraalVM would be out of scope of this project. It would represent the throrough 
2nd step of the compiler verification research thread \cite[pp. 5]{CompilerOptimization} -- similar to Alive2's \cite{Alive2} solution on formal 
verification (See \ref{sec:Alive}); perhaps a future direction in Veriopt.

\todo[inline]{Describe the sections}