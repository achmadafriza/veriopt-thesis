\section{Project Plan}
% From the Project Proposal marking sheet:
%   - A well justified, comprehensive, and feasible list of milestones.
%   (with resources and duration).
%   - A complete and accurate risk and/or ethics assessment.

This project would require several milestones to be completed:
\begin{enumerate}
    \item Understanding Isabelle System (September - October);
    \item Creating a Proof of Concept (PoC) for the Automated Testing Framework (September - October);
    \item Refining the PoC to be usable in a production setting (November - February).
\end{enumerate}

However, due to the nature of the possible solutions of the framework, it could mean that the details and deadline 
of the milestones could shift dramatically, as each of the solutions differs in their concept. Therefore, 
each of the milestones would depict several common goals that need to be achieved in order to 
provide value to the project. Furthermore, the project should allocate several months as a contingency for 
bug-fixing and unexpected delays (March-May, about 30\% of the total project timeline).

Furthermore, it would require a considerable amount of time for me to complete the requirements for the course,
which would be done alongside the project milestones, namely:
\begin{enumerate}
    \item Progress Seminar (Due 9 October 2023);
    \item Conference Paper (Due 9 May 2024);
    \item Poster \& Demonstration (Due 17 May 2024);
    \item Thesis Submission (Due 3 June 2024).
\end{enumerate}

\subsection{Milestones}
\label{sec:Milestones}
% Break your project down into component sub-goals and estimate how long
% you expect each to take.

\subsubsection{Milestone 1: Understanding Isabelle System}

Understanding Isabelle's system infrastructure would require the project to explore:
\begin{enumerate}
    \item Isabelle Server (September)
    \item Isabelle/Scala (October, \emph{if needed})
\end{enumerate}

Exploring the possible solutions can be done alongside Milestone 2.

\subsubsection{Milestone 2: Creating a Proof of Concept}

Creating a PoC would be done in several steps:
\begin{enumerate}
    \item Creating a PoC with no user interaction (September);
          
          This would allow the feasibility of possible solutions to be evaluated before the Progress Seminar.

    \item Generating evaluation parameters and testing the PoC (October);
\end{enumerate}

\subsubsection{Milestone 3: Refining the Proof of Concept}

Refining the PoC would be done in a similar manner to refining production services. It would be done in these steps:
\begin{enumerate}
    \item Defining the system architecture for a scalable service (November)
    \item Extending the PoC into RESTful APIs and allowing concurrent access (November-December);
    \item Implementing thorough error handling and logging (December);
    \item Deployment of the service (January);
    \item Integration tests and Stress tests (January-February).
\end{enumerate}

\subsection{Timeline}
\todo[inline]{create a gantt chart for the milestones}
\lipsum[1-2]

\subsection{Risk Assessment}
% What, if any, risks does your project pose.
% What risks are there to your proposed timeline (technical or otherwise).

There are several risks that need to be considered:
\begin{itemize}
    \item Complexity of utilizing Isabelle is understated (Critical);
    
          Mitigation:
          \begin{itemize}
            \item Allocating several months as a contingency plan for the timeline (min. 25\%);
            \item Frequent consultation with supervisors.
          \end{itemize}

    \item The solution does not meet the key system requirements (High);
    
          Mitigation: Frequent discussion about the system specifications with supervisors.

    \item Problems with personal devices, i.e. short-circuited motherboard (Medium);
    
          Mitigation:
          \begin{itemize}
            \item Frequent backups to the cloud;
            \item Utilizing UQ-provided labs for development.
          \end{itemize}

    \item OHS risks (Low):
    
          Mitigation: The majority of the development would be done in a low-risk setting.
\end{itemize}

\subsection{Ethics Assessment}
% Are there any ethical considerations involved in your project.
% e.g. do you make use of sensitive or personal information.

Minimal ethical considerations are present in the project, due to the scope of the project only involving 
software.